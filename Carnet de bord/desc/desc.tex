\section{Descriptif de la recherche documentaire}
Du fait de la nature de notre projet qui est qualifié comme projet de développement, la recherche documentaire est moins notable qu'un sujet classique de recherche. Néanmoins, nous nous sommes servis de l'outil \textit{SUper} sur le site de la bibliothèque universitaire Pierre et Marie Curie\footnote{\url{http://www.bupmc.upmc.fr/fr/index.html}} sur les mots clés de départ afin d'obtenir quelques pistes dans notre développement. Les documents ainsi trouvés se sont avérés très généralistes et pas forcément pertinents par rapport à nos objectifs. Plus généralement, nous nous sommes principalement servis des documentations \og officielles \fg{} pour nous appuyer dans notre programmation, à savoir la documentation du langage Java mais aussi celle du constructeur du drone pour développer notre programme de pilotage.

En ce qui concerne notre jeu mobile, nous nous sommes beaucoup appuyés sur des cours en ligne de programmation mobile \android{} pour nous guider. Nous nous sommes aussi intéressés sur la possibilité de générer des niveaux à partir d'un morceau de musique par analyse audio, peu de résultats exploitables ont été trouvés.

Finalement, l'outil \textit{SUper} a été utile comme mise en route pour notre recherche mais c'est surtout les documentations officielles et supports de programmation qui nous ont été profitables pour mener à bien ce projet.