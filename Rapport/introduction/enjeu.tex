\newpage
\subsection{Enjeu}
Il va principalement s'agir dans ce projet de réussir à manipuler un drone, le jeu ne sert en effet que de prétexte pour l'utilisation de ce dernier. Ainsi, la forme finale de l'application mobile peut changer selon les limites et difficultés rencontrées; l'un des objectifs est de présenter une démonstration du travail effectué à l'occasion de la \emph{Fête de la science}\footnote{du 12 au 15 octobre 2017}. Une étude des ressources à disposition dont le kit de développement logiciel\footnote{SDK : Software Development Kit} fourni par le constructeur sera donc nécessaire; le modèle du drone utilisé est le Parrot BEBOP 2\footnote{\url{https://www.parrot.com/fr/Drones/Parrot-bebop-2}}.

\paragraph{}
Le développement est décomposé en 3 grandes parties que nous présenterons plus en détails dans ce rapport :
\begin{enumerate}
\item Conception du jeu mobile
\item Développement du serveur de jeu multijoueur
\item Manipulation du drone
\end{enumerate}

\paragraph{}
Comme on peut le voir, c'est un environnement client-serveur que nous allons mettre en place. En effet, les joueurs, à partir de leur application mobile, représentent les clients et se connecteront à un serveur hébergé sur un ordinateur dont le rôle sera de centraliser les scores des joueurs. A partir de ces données, le serveur communiquera avec le drone pour le piloter vers l'un des joueurs. 