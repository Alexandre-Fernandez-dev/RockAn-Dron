\subsection{Sujet}
Nous présentons ci-dessous le sujet du projet tel qu'il est décrit aux étudiants lors de la sélection :

\paragraph{}
Dans le cadre des activités robotiques au sein du M2 STL, nous aurions besoin d’explorer la programmation sur les drones. A l’heure actuelle, toutes les réalisations robotiques du M2 STL se basent sur des plateformes mobiles terrestres. Il est souhaitable d’explorer d’autres possibilités de support. Nous proposons de nous attaquer à ce projet sous un angle "Kid games", en récoltant des données de jeux multi-joueurs et en envoyant des commandes de déplacement selon ces données sur un drone baptisé "RockAn’Dron".

\paragraph{}
Le premier de tels jeux serait un Tambour-Hero où de multiples joueurs jouent en parallèle un jeu de rythme sur des smartphones Android en mode "client". Les résultats de chaque joueur sont évalués dynamiquement sur chaque support "client" avant d’être synchronisés sur un smartphone Android en mode "serveur". Les évaluations de chaque joueur doivent refléter sa capacité à imiter un batteur dans un morceau de musique folle. A chaque synchronisation sur le "serveur"(de l’ordre de la seconde par exemple), le smartphone "serveur" décide le gagnant actuel et envoie le drone faire un pas vers le gagnant en question. Le jeu termine quand un des joueurs arrive à attirer vers lui le drone, porteur de gros paquets de bonbons, ou de nouilles, voire d’autres choses selon des envies du moment.

\paragraph{}
Ce PSTL s’adresse à des étudiants en M1 STL curieux des réalisations robotiques ouvertes au grand publique de type "tout en kit"\footnote{Les étudiants curieux des réalisations robotiques à partir de (quasiment) rien pourraient s’intéresser au PSTL "Prototype de robot mobile".}. Dépendant du niveau de chaque participant au PSTL, on commencerait avec une partie de programmation purement Android, puis, on réfléchirait à l’intégrer dans les actionneurs du drone ; ou l’inverse ; voire on ferait des choses complètement imprévues. Une réalisation, cependant, est attendue au terme du PSTL, avec poster de présentation et montage de vidéo-clip.