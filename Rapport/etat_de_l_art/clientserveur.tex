\subsection{Modèle client-serveur}

Dans notre cas, le modèle client-serveur, pour se faire communiquer nos différents programmes au sein d'un réseau local, était parfaitement adapté à la situation. Ce type d'environnement tout à fait classique dans le domaine de la programmation réseau a été étroitement étudié au cours de cette année universitaire, en enseignement de \textit{Programmation répartie} ou encore de \textit{Programmation concurrente, réactive et répartie} par exemple.