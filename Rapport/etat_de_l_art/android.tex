\subsection{Programmation mobile \textit{Android}}

De plus en plus d'applications mobiles ont été développées au cours de ces dernières années, notamment depuis l'arrivée sur le marché des smartphones qui ont contribué à l'explosion du marché des applications mobiles. Le choix du système d'exploitation sur lequel nous allions développer notre jeu mobile s'est porté sur \textit{Android}\footnote{Système d'exploitation mobile basé sur le noyau Linux développé par \textit{Google}}. Cette décision était pour nous la plus pertinente au vu de la domination de l'OS sur le marché.

Nous nous sommes donc intéressés à la programmation mobile sous \textit{Android} et à sa SDK\footnote{Kit de développement logiciel}. Le code associé est en Java.

\paragraph{}
De plus, nous nous sommes servis du framework \textit{libGDX}\footnote{\url{https://github.com/libgdx/libgdx}} qui est une interface de programmation Java multi-plateformes pour développer notre jeu de rythme. Cela étant décidé, il nous fallait trouver un moyen de lier cette partie au reste.